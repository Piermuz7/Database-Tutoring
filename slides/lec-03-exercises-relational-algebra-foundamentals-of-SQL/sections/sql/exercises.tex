\begin{frame}{Esercizio 4.1}
         Ordinare i seguenti domini in base al valore massimo rappresentabile, supponendo che \texttt{integer} abbia una rappresentazione a 32 bit e \texttt{smallint} a 16 bit:
        \begin{enumerate}
            \item \texttt{numeric(12,4)}
            \item \texttt{decimal(10)}
            \item \texttt{decimal(9)}
            \item \texttt{integer}
            \item \texttt{smallint}
            \item \texttt{decimal(6,1)}
        \end{enumerate}
\end{frame}

\begin{frame}{Esercizio 4.1}
    \textbf{Soluzione:}
    \begin{table}[h]
        \centering
        \begin{tabular}{|c|c|}
            \hline
            Dominio & Valore Massimo \\
            \hline
            \texttt{decimal(10)} & 9999999999 \\
            \texttt{integer} & 4294967296 \\
            \texttt{decimal(9)} & 999999999 \\
            \texttt{numeric(12,4)} & 99999999.9999 \\
            \texttt{decimal(6,1)} & 99999.9 \\
            \texttt{smallint} & 65536 \\
            \hline
        \end{tabular}
    \end{table}
\end{frame}
%%%%%%%%%%%%%%%%
\begin{frame}{Esercizio 4.7}
    Con riferimento ad una relazione \texttt{PROFESSORI(CF, Nome, Et\`a, Qualifica)}, scrivere le interrogazioni SQL che calcolano l'et\`a media dei professori di ciascuna qualifica, nei due casi seguenti:
    \begin{enumerate}
        \item se l'et\`a non \`e nota si usa per essa il valore nullo;
        \item se l'et\`a non \`e nota si usa per essa il valore 0.
    \end{enumerate}
\end{frame}

\begin{frame}{Esercizio 4.7}
    \textbf{Soluzione:}
    Le funzioni aggregative escludono dalla valutazione le ennuple con valori nulli:
    \vspace{1em}
    
    \texttt{SELECT Qualifica, AVG(Et\`a) AS Et\`aMedia\\FROM Professori\\GROUP BY Qualifica;}
\end{frame}

\begin{frame}{Esercizio 4.7}
    \textbf{Soluzione:}
    \`E necessario escludere esplicitamente dal calcolo della media le ennuple con il valore che denota l'informazione incompleta:
    \vspace{1em}
    
    \texttt{SELECT Qualifica, AVG(Et\`a) AS Et\`aMedia\\FROM Professori\\WHERE Et\`a <> 0\\GROUP BY Qualifica;}
\end{frame}
%%%%%%%%%%%%%%%%
\begin{frame}{Esercizio 4.14}
    Dato il seguente schema:
    \begin{itemize}
        \item AEROPORTO(Citt\`a, Nazione,NumPiste)
        \item VOLO(IdVolo,GiornoSett,Citt\`aPart,OraPart,
        Citt\`aArr,OraArr,TipoAereo)
        \item AEREO(TipoAereo,NumPasseggeri,QtaMerci)
    \end{itemize}
    \vspace{1em}
    
    \textbf{Le citt\`a con un aeroporto di cui non \`e noto il numero di piste.}
\end{frame}

\begin{frame}{Esercizio 4.14}
    \textbf{Soluzione:}
    \vspace{1em}
    
    \texttt{SELECT DISTINCT Citt\`a\\FROM AEROPORTO\\WHERE NumPiste IS NULL;}
    \end{frame}
%%%%%%%%%%%%%%%%%%%
\begin{frame}{Esercizio 4.14}
    Dato il seguente schema:
    \begin{itemize}
        \item AEROPORTO(Citt\`a, Nazione,NumPiste)
        \item VOLO(IdVolo,GiornoSett,Citt\`aPart,OraPart,
        Citt\`aArr,OraArr,TipoAereo)
        \item AEREO(TipoAereo,NumPasseggeri,QtaMerci)
    \end{itemize}
    \vspace{1em}
    
    \textbf{Le nazioni da cui parte e arriva il volo con codice AZ274.}
\end{frame}

\begin{frame}{Esercizio 4.14}
    \textbf{Soluzione:}
    \vspace{1em}
    
    \texttt{SELECT A1.Nazione, A2.Nazione\\FROM AEROPORTO AS A1\\ JOIN VOLO ON A1.Citt\`a = Citt\`aArr \\JOIN AEROPORTO AS A2 ON Citt\`aPart = A2.Citt\`a\\WHERE IdVolo = `AZ274';
}
    \end{frame}
%%%%%%%%%%%%%%%%%%%
\begin{frame}{Esercizio 4.14}
    Dato il seguente schema:
    \begin{itemize}
        \item AEROPORTO(Citt\`a, Nazione,NumPiste)
        \item VOLO(IdVolo,GiornoSett,Citt\`aPart,OraPart,
        Citt\`aArr,OraArr,TipoAereo)
        \item AEREO(TipoAereo,NumPasseggeri,QtaMerci)
    \end{itemize}
    \vspace{1em}
    
    \textbf{Le citt\`a con un aeroporto di cui non \`e noto il numero di piste.}
\end{frame}

\begin{frame}{Esercizio 4.14}
    \textbf{Soluzione:}
    \vspace{1em}
    
    \texttt{SELECT DISTINCT Citt\`a\\FROM AEROPORTO\\WHERE NumPiste IS NULL;}
    \end{frame}
%%%%%%%%%%%%%%%%%%%
\begin{frame}{Esercizio 4.14}
    Dato il seguente schema:
    \begin{itemize}
        \item AEROPORTO(Citt\`a, Nazione,NumPiste)
        \item VOLO(IdVolo,GiornoSett,Citt\`aPart,OraPart,
        Citt\`aArr,OraArr,TipoAereo)
        \item AEREO(TipoAereo,NumPasseggeri,QtaMerci)
    \end{itemize}
    \vspace{1em}
    
    \textbf{I tipi di aereo usati nei voli che partono da Torino}
\end{frame}

\begin{frame}{Esercizio 4.14}
    \textbf{Soluzione:}
    \vspace{1em}
    
    \texttt{SELECT TipoAereo\\FROM VOLO\\WHERE Citt\`aPart=`Torino'}
    \end{frame}
%%%%%%%%%%%%%%%%%%%
\begin{frame}{Esercizio 4.14}
    Dato il seguente schema:
    \begin{itemize}
        \item AEROPORTO(Citt\`a, Nazione,NumPiste)
        \item VOLO(IdVolo,GiornoSett,Citt\`aPart,OraPart,
        Citt\`aArr,OraArr,TipoAereo)
        \item AEREO(TipoAereo,NumPasseggeri,QtaMerci)
    \end{itemize}
    \vspace{1em}
    
    \textbf{Le citt\`a con un aeroporto di cui non \`e noto il numero di piste.}
\end{frame}

\begin{frame}{Esercizio 4.14}
    \textbf{Soluzione:}
    \vspace{1em}
    
    \texttt{SELECT DISTINCT Citt\`a\\FROM AEROPORTO\\WHERE NumPiste IS NULL;}
    \end{frame}
%%%%%%%%%%%%%%%%%%%
\begin{frame}{Esercizio 4.14}
    Dato il seguente schema:
    \begin{itemize}
        \item AEROPORTO(Citt\`a, Nazione,NumPiste)
        \item VOLO(IdVolo,GiornoSett,Citt\`aPart,OraPart,
        Citt\`aArr,OraArr,TipoAereo)
        \item AEREO(TipoAereo,NumPasseggeri,QtaMerci)
    \end{itemize}
    \vspace{1em}
    
    \textbf{I tipi di aereo e il corrispondente numero di passeggeri per i tipi di aereo usati nei voli che partono da Torino. Se la descrizione dell`aereo non \`e disponibile, visualizzare solamente il tipo;}
\end{frame}

\begin{frame}{Esercizio 4.14}
    \textbf{Soluzione:}
    \vspace{1em}
    
    \texttt{SELECT VOLO.TipoAereo, NumPasseggeri\\FROM VOLO LEFT JOIN AEREO ON VOLO.TipoAereo=aereo.TipoAereo\\WHERE Citt\`aPart= `Torino'}
    \end{frame}
