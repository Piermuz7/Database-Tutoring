\begin{frame}{Identificatori}
    \onslide<1->Gli \textbf{Identificatori} (nomi di tabelle e di attributi) sono costituiti da sequenze di caratteri: devono iniziare con una lettera e possono anche contenere il carattere \_ .
    \newline
    \onslide<2->\\ Il nome di un attributo (colonna di una tabella) \`e identificato per mezzo della notazione:
    \[ NomeTabella.NomeAttributo \]
    \onslide<2->Il nome della tabella pu\`o essere omesso se non ci sono ambiguit\`a nell'identificazione dell'attributo.
\end{frame}
%
\begin{frame}[allowframebreaks]{Tipi di Dati}
    Ogni colonna di una tabella di database deve avere un nome e un tipo di dati.
    \centering
    \begin{tabular}{|c|c|}
        \hline
        \rowcolor{cyan!30}Tipo di dato & Descrizione \\
        \hline
        CHARACTER(n) & Stringa di caratteri. Lunghezza fissa n \\ \hline
        VARCHAR(n) & Stringa di caratteri. Lunghezza variabile. Lunghezza massima n \\ \hline
        BINARY(n) & Stringa binaria. Lunghezza fissa n \\ \hline
        BOOLEAN & Memorizza i valori TRUE o FALSE \\ \hline
        VARBINARY(n) & Stringa binaria. Lunghezza variabile. Lunghezza massima n \\ \hline
        INTEGER(p) & Numero intero (senza decimali). Precisione p \\ \hline
        SMALLINT & Numero intero (senza decimali). Precisione 5 \\ \hline
        INTEGER	& Numero intero (senza decimali). Precisione 10 \\ \hline
        BIGINT & Numero intero (senza decimali). Precisione 19 \\ \hline
        DECIMAL(p,s) & Numero decimale con precisione p e s cifre decimali \\ \hline
        NUMERIC(p,s) & Numero decimale con precisione p e s cifre decimali \\ \hline
        \end{tabular}

        \centering
    \begin{tabular}{|c|c|}
        \hline
        \rowcolor{cyan!30}Tipo di dato & Descrizione \\
        \hline
        FLOAT(p) & Numero reale con mantissa di precisione p \\ \hline
        REAL & Numero reale con mantissa di precisione 7 \\ \hline
        FLOAT & Numero reale con mantissa di precisione 16 \\ \hline
        DOUBLE PRECISION & Numero reale con mantissa di precisione 16 \\ \hline
        DATE & Memorizza i valori di anno, mese e giorno \\ \hline
        TIME & Memorizza i valori di ore, minuti e secondi \\ \hline
        TIMESTAMP & Memorizza i valori di anno, mese, giorno, ora, minuto e secondo \\ \hline
        ARRAY & un insieme ordinato di n elementi \\ \hline
        XML & Memorizza dati XML \\ \hline
        \end{tabular}
\end{frame}