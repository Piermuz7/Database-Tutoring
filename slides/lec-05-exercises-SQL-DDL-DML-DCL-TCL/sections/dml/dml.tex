\begin{frame}{Manipolare i dati}
    \onslide<1->{I valori degli attributi nelle righe della tabella possono essere:
    \begin{itemize}
        \item inseriti (\textbf{INSERT})
        \item aggiornati (\textbf{UPDATE})
        \item cancellati (\textbf{DELETE})
    \end{itemize}}
    \onslide<2->{
    \begin{alertblock}{La clausola WHERE nel DML}
        \`E importante notare che nei comandi \textbf{UPDATE} e \textbf{DELETE} compare la clausola \textbf{WHERE} per effetto della quale \`e possibile operare su molte righe, anzich\'e su una sola riga per volta.
    \end{alertblock}
    }
    \onslide<3>{
    \begin{alertblock}{ATTENZIONE al WHERE}
        La clausola \textbf{WHERE} specifica quali records dovrebbero essere aggiornati. Se essa viene omessa, verranno aggiornati tutti i record presenti nella tabella!
    \end{alertblock}
    }
\end{frame}
%
\begin{frame}[fragile]{DML: INSERT}
Il comando \textbf{INSERT INTO} viene usato per inserire nuovi records nella tabella.

\`E possibile utilizzarlo in 2 modi:
\begin{enumerate}
\item Specificando i nomi delle colonne e dei valori da inserire per tali:
\begin{lstlisting}
INSERT INTO nomeTabella (colonna1, colonna2, colonna3, ...)
VALUES (valore1, valore2, valore3, ...);
\end{lstlisting}
\item Senza specificare i nomi delle colonne quando si vogliono aggiungere i valori per tutte le colonne della tabella.
\begin{lstlisting}
INSERT INTO nomeTabella
VALUES (valore1, valore2, valore3, ...);
\end{lstlisting}
\end{enumerate}
\end{frame}
%
\begin{frame}[fragile]{INSERT: Impiegati}
\begin{itemize}
\item Inserimento di \textbf{una riga} nella tabella Impiegati:
\begin{lstlisting}
INSERT INTO Impiegati
(ID, Nome, Cognome, Residenza, Stipendio, Dipartimento)
VALUES(20, `Mario', `Rossi', `Camerino', 99999, `R&S');
\end{lstlisting}
\item Inserimento di \textbf{pi\`u righe} nella tabella Impiegati:
\begin{lstlisting}
INSERT INTO Impiegati
(ID, Nome, Cognome, Residenza, Stipendio, Dipartimento)
VALUES
(20, `Mario', `Rossi', `Camerino', 99999, `R&S'),
(77, `Luigi', `Verdi', `Castelraimondo', 1234, `Magazzino');
\end{lstlisting}
\end{itemize}
\end{frame}
%
\begin{frame}[fragile]{DML: UPDATE}
Il comando \textbf{UPDATE} viene usato per modificare records esistenti nella tabella.

\begin{lstlisting}
UPDATE nomeTabella
SET colonna1 = valore1, colonna2 = valore2, ...
WHERE condizione;
\end{lstlisting}
\end{frame}
%
\begin{frame}[fragile]{UPDATE: Impiegati}
Assegnamento del dipendente con ID=20 al dipartimento \textit{Produzione}:
\pause
\begin{lstlisting}
UPDATE Impiegati
SET Dipartimento = `Prod'
WHERE ID = 20;
\end{lstlisting}
\end{frame}
%
\begin{frame}[fragile]{UPDATE: Impiegati}
Aumento stipendio del 5\% ai dipendenti del dipartimento \textit{Produzione}:
\pause
\begin{lstlisting}
UPDATE Impiegati
SET Stipendio = Stipendio * 1.05
WHERE Dipartimento = `Prod';
\end{lstlisting}
\end{frame}
%
\begin{frame}[fragile]{DML: DELETE}
Il comando \textbf{DELETE} viene usato per cancellare records nella tabella.

\begin{lstlisting}
DELETE FROM nomeTabella WHERE condizione;
\end{lstlisting}
\end{frame}
%
\begin{frame}[fragile]{DELETE: Impiegati}
Cancellazione del dipendente con ID = 20 dalla tabella \textit{Impiegati}:
\pause
\begin{lstlisting}
DELETE FROM Impiegati
WHERE ID = 20;
\end{lstlisting}
\end{frame}
%
\begin{frame}[fragile]{DELETE: Impiegati}
Cancellazione dei dipendenti del dipartimento di Ricerca \& Sviluppo:
\pause
\begin{lstlisting}
DELETE FROM Impiegati
WHERE Dipartimento = `R&S';
\end{lstlisting}
\end{frame}
%
\begin{frame}[fragile]{DELETE: Impiegati}
Cosa produce il seguente comando?
\begin{lstlisting}
DELETE FROM Impiegati;
\end{lstlisting}
\pause
\begin{alertblock}{DELETE senza WHERE}
Manca la clausola \textbf{WHERE}: il comando svuota la tabella \textit{Impiegati} dai dati, ma rimane la sua struttura.
\end{alertblock}
\pause
{\begin{block}{Svuotare le tabelle}
Per svuotare una tabella \`e possibile utilizzare DELETE FROM nomeTabella. Tuttavia, la maniera pi\`u corretta \`e utilizzare il comando \textbf{TRUNCATE TABLE} nomeTabella, il quale elimina i dati ma mantiene la struttura della tabella.
\end{block}
\begin{lstlisting}
TRUNCATE TABLE Impiegati;
\end{lstlisting}
}
\end{frame}
%