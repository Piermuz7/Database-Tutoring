\begin{frame}[fragile]{Definire i permessi nel database}
\onslide<1->{I comandi di tipo \textbf{Data Control Language (DCL)} hanno lo scopo di garantire la sicurezza agli oggetti del database.}
\newline
\\\onslide<2->{Nei DBMS SQL ogni operazione deve essere autorizzata, ovvero l'utente che esegue l'operazione deve avere i privilegi necessari.}
\newline
\\\onslide<3->{Il database administrator (DBA) pu\`o:
\begin{itemize}
    \item concedere, con il comando \textbf{GRANT}
    \item o revocare, con il comando \textbf{REVOKE},
\end{itemize}
il diritto di eseguire certe azioni su determinati oggetti del database a specifici (o tutti gli) utenti.}
\end{frame}
%
\begin{frame}[fragile]{DCL: GRANT}
Il comando \textbf{GRANT} concede i permessi, specificando il tipo di accesso, per le tabelle sulle quali \`e consentito l'accesso e l'elenco degli utenti ai quali \`e permesso di accedere.
\newline
\\ Il tipo di accesso pu\`o riguardare, per esempio:
\begin{itemize}[<+->]
    \item il diritto di modifica della struttura della tabella con l'aggiunta di nuove colonne;
    \item oppure di modifica dei dati contenuti nella tabella;
    \item oppure l'uso del comando SELECT
    \item \ldots
\end{itemize}
\end{frame}
%
\begin{frame}[fragile]{DCL: GRANT - privilegi per Schemi}
\begin{itemize}
\item Sintassi di GRANT per concedere privilegi su schemi:
\begin{lstlisting}
GRANT < lista_privilegi >
ON SCHEMA < nome_schema >
TO { < lista_utenti e gruppi |$\vert$| PUBLIC > }
[ WITH GRANT OPTION ];
\end{lstlisting}
\item Possibili privilegi:
\\\textbf{CREATEIN}: per creare oggetti (tabelle, viste) nello schema;

\textbf{ALTERIN}:~~~per modificare la struttura di tabelle dello schema;

\textbf{DROPIN}:~~~~per eliminare oggetti dallo schema.
\end{itemize}
\end{frame}
%
\begin{frame}[fragile]{Esempio DCL: GRANT - privilegi per Schemi}
Il seguente comando DCL consente agli utenti USR7 e USR42 di creare e modificare oggetti nello schema DB12345:
\begin{lstlisting}
GRANT CREATEIN, ALTERIN
ON SCHEMA DB12345
TO USR7, USR42;
\end{lstlisting}
\end{frame}
%
\begin{frame}[fragile]{DCL: GRANT - privilegi per Tabelle e Viste}
Sintassi di GRANT per concedere privilegi su tabelle e viste:
\begin{lstlisting}
GRANT < ALL |$\vert$| lista_privilegi >
ON [ TABLE ] < nome_tabella >
TO { < lista_utenti e gruppi |$\vert$| PUBLIC > }
[ WITH GRANT OPTION ];
\end{lstlisting}
\end{frame}
%
\begin{frame}[fragile]{Esempio DCL: GRANT - privilegi per Tabelle}
Per concedere il diritto di modifica sulla tabella \textit{Impiegati} agli utenti User1 e User2:
\begin{lstlisting}
GRANT UPDATE
ON Impiegati
TO User1, User2;
\end{lstlisting}
\end{frame}
%
\begin{frame}[fragile]{DCL: REVOKE}
La revoca dei permessi con annullamento dei diritti di accesso viene effettuata con il comando \textbf{REVOKE} che ha una sintassi analoga al GRANT.
\end{frame}
\begin{frame}[fragile]{DCL: REVOKE - privilegi da Tabelle e Viste}
Sintassi di REVOKE per rimuovere privilegi da tabella e viste:
\begin{lstlisting}
REVOKE < ALL |$\vert$| lista_privilegi >
ON [ TABLE ] < nome_tabella >
FROM { < lista_utenti e gruppi |$\vert$| PUBLIC > };
\end{lstlisting}
\end{frame}
%
\begin{frame}[fragile]{Esempio DCL: REVOKE - privilegi da Tabelle}
Per revocare il permesso di modifica sulla tabella \textit{Impiegati} agli utenti User1 e User2:
\begin{lstlisting}
REVOKE UPDATE
ON Impiegati
FROM User1, User2;
\end{lstlisting}
\end{frame}
%
\begin{frame}[fragile]{DCL: Privilegi SQL}
I permessi che possono essere concessi (o revocati) agli utenti sono indicati con le seguenti parole chiave, le quali vanno specificate dopo GRANT o REVOKE:
\newline
\\\textbf{ALTER}:~~~~~ per aggiungere o eliminare colonne oppure per modificare tipi di dati;
\\\textbf{DELETE}:~~~ per eliminare righe dalle tabelle;
\\\textbf{INDEX}:~~~~~ per creare indici;
\\\textbf{INSERT}:~~~~per inserire nuove righe nelle tabelle;
\\\textbf{SELECT}:~~~ per ritrovare i dati  dalle tabelle;
\\\textbf{UPDATE}:~~ per cambiare i valori contenuti nelle tabelle;
\\\textbf{ALL}:~~~~~~~~~ per tutti i permessi precedenti.
\end{frame}